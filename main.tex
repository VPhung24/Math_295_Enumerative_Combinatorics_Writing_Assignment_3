% --------------------------------------------------------------
% This is all preamble stuff that you don't have to worry about.
% Head down to where it says "Start here"
% --------------------------------------------------------------
 
\documentclass[12pt]{article}
 
\usepackage[margin=1in]{geometry} 
\usepackage{amsmath,amsthm,amssymb}
 
\newcommand{\N}{\mathbb{N}}
\newcommand{\Z}{\mathbb{Z}}
 
\newenvironment{theorem}[2][Theorem]{\begin{trivlist}
\item[\hskip \labelsep {\bfseries #1}\hskip \labelsep {\bfseries #2.}]}{\end{trivlist}}
\newenvironment{lemma}[2][Lemma]{\begin{trivlist}
\item[\hskip \labelsep {\bfseries #1}\hskip \labelsep {\bfseries #2.}]}{\end{trivlist}}
\newenvironment{exercise}[2][Exercise]{\begin{trivlist}
\item[\hskip \labelsep {\bfseries #1}\hskip \labelsep {\bfseries #2.}]}{\end{trivlist}}
\newenvironment{reflection}[2][Reflection]{\begin{trivlist}
\item[\hskip \labelsep {\bfseries #1}\hskip \labelsep {\bfseries #2.}]}{\end{trivlist}}
\newenvironment{proposition}[2][Proposition]{\begin{trivlist}
\item[\hskip \labelsep {\bfseries #1}\hskip \labelsep {\bfseries #2.}]}{\end{trivlist}}
\newenvironment{corollary}[2][Corollary]{\begin{trivlist}
\item[\hskip \labelsep {\bfseries #1}\hskip \labelsep {\bfseries #2.}]}{\end{trivlist}}
 
\begin{document}
 
% --------------------------------------------------------------
%                         Start here
% --------------------------------------------------------------
 
%\renewcommand{\qedsymbol}{\filledbox}
 
\title{Writing Assignment 3}%replace X with the appropriate number
\author{Vivian Phung\\ \\ %replace with your name
MATH 295 - Enumerative Combinatorics\\With Professor Amy Myers} 

\maketitle
 

\textbf{Section 7.1 Exercise 6(b):} Find a recurrence relation for the number of $n$-digit ternary sequences with no pair of consecutive 1s. \\

As a ternary sequence, the digit choices are $0$, $1$, and $2$. The initial conditions for this recursive relations is $a_{0} = 1$ and $a_{1} = 3$ because
\begin{itemize}
  \item $a_{0} = 1$: An empty sequence of $0$ digits occurs $1$ time (when $n = 0$)
  \item $a_{1} = 3$: A sequence of 1 can be either 0, 1, or 2.
\end{itemize} \\

To enumerate $n$-digit ternary sequences with no pair of consecutive $1$s, let's look at the possible cases. If there are $n$ digits in the sequence, the first digit can be
\begin{itemize}
  \item 0; There are $a_{n-1}$ possible sequences that can be placed after 0
  \item 1; To definitively not get any consecutive 1s, the second digit in the sequence can be
  \begin{itemize}
      \item 0; There are $a_{n-2}$ possible sequences that can be placed after 10
      \item 2; There are $a_{n-2}$ possible sequences that can be placed after 12
    \end{itemize}
  \item 2; There are $a_{n-1}$ possible sequences that can be placed after 2
\end{itemize} \\

Since these cases are disjoint, add the values to find the recursive relation equation:
\begin{align*}
a_{n} &= a_{n-1} + a_{n-2} + a_{n-2} + a_{n-1} \\
a_{n} &= 2 \cdot a_{n - 1} +  2 \cdot a_{n - 2}
\end{align*}


Thus, the recurrence relation for the number of $n$-digit ternary sequences with no pair of consecutive $1$s is $a_{n} = 2 \cdot a_{n - 1} +  2 \cdot a_{n - 2}$ where $a_{0} = 1$ and $a_{1} = 3$.\\
% --------------------------------------------------------------
%     You don't have to mess with anything below this line.
% --------------------------------------------------------------
 
\end{document}